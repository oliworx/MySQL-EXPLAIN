% Literaturliste soll im Inhaltsverzeichnis auftauchen
\newpage
\addcontentsline{toc}{section}{Literatur}


  \begin{thebibliography}{}

%  Monographien:
%1.  Familienname des Verfassers
%2.  Vorname des Verfassers, vorzugsweise abgekürzt
%3.  Erscheinungsjahr in Klammern
%4.  vollständiger Titel des Werks
%5.  Auflage (Aufl.), wenn es sich nicht um die erste Auflage handelt
%6.  Erscheinungsort, der grundsätzlich dem Verlagsort entspricht (mehr als 3 Orte sind nicht zu nennen; stattdessen: u.a.) und / oder Verlag
%7.  Erscheinungsjahr
%8.  Handelt es sich bei der Literaturquelle um eine Dissertation, ist vor 5. die Abkürzung „Diss.“ Einzufügen

% Beispiele:
%Scholz, C. (2000): Personalmanagement, 5. Aufl., München 2000
%Lüdenbach, N., Hoffmann, W.-D. (2003): Haufe IAS-Kommentar, Freiburg 2003
%Ruhnke, K. (1995): Konzernbuchführung, Diss., Düsseldorf 1995

    \bibitem{Schwartz, B. (2009)} Baron Schwartz, Peter Zaitsev, Vadim Tkachenko, Jeremy D. Zawodny, Arjen Lentz, Derek J. Balling
     High Performance MySQL. Optimierung, Datensicherung, Replikation \& Lastverteilung, 2. Auflage,  O'Reilly Verlag, 2009

    \bibitem {Sauer, H. (1998)}  Relationale Datenbanken, Theorie und Praxis, 4. Auflage, Addison Wesley Longman Verlag, 1998

    \bibitem {Bradford, R. (2011)} Effective MySQL: Optimizing SQL Statements, 1. Auflage, McGraw-Hill Osborne Media, 2011

  \end{thebibliography}


% Literaturliste endgueltig anzeigen
\bibliography{literatur} 