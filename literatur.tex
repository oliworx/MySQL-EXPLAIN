% Literaturliste soll im Inhaltsverzeichnis auftauchen
\newpage
\addcontentsline{toc}{section}{Literatur}
  \begin{thebibliography}{}

%  Monographien:
%1.  Familienname des Verfassers
%2.  Vorname des Verfassers, vorzugsweise abgekürzt
%3.  Erscheinungsjahr in Klammern
%4.  vollständiger Titel des Werks
%5.  Auflage (Aufl.), wenn es sich nicht um die erste Auflage handelt
%6.  Erscheinungsort, der grundsätzlich dem Verlagsort entspricht (mehr als 3 Orte sind nicht zu nennen; stattdessen: u.a.) und / oder Verlag
%7.  Erscheinungsjahr
%8.  Handelt es sich bei der Literaturquelle um eine Dissertation, ist vor 5. die Abkürzung „Diss.“ Einzufügen

% Beispiele:
% Scholz, C. (2000): Personalmanagement, 5. Aufl., München 2000
% Lüdenbach, N., Hoffmann, W.-D. (2003): Haufe IAS-Kommentar, Freiburg 2003
% Ruhnke, K. (1995): Konzernbuchführung, Diss., Düsseldorf 1995

    \bibitem{Bradford2011} Bradford, R. (2011) {\sl Effective MySQL: Optimizing SQL Statements}, Oracle Press/McGraw-Hill Osborne Media, 2011
    \bibitem{Sauer1998} Sauer, H. (1998) {\sl Relationale Datenbanken, Theorie und Praxis}, 4. Auflage, Addison Wesley Longman Verlag, 1998
    \bibitem{Schwartz2009} Schwartz, B., Zaitsev, P., Tkachenko, V., Zawodny, J.D.,Lentz, A., Balling, D.J. (2009) 
        {\sl High Performance MySQL. Optimierung, Datensicherung, Replikation \& Lastverteilung}, 2. Auflage,  O'Reilly Verlag, 2009
	
	Internetquellen:
	\bibitem{refman1}ORACLE MySQL Documentation (2014): {\sl Optimizing Queries with EXPLAIN}. URL: 
	\url{http://dev.mysql.com/doc/refman/5.6/en/using-explain.html}, Abruf am 28.6.2014
    \bibitem{refman2}ORACLE MySQL Documentation (2014): {\sl EXPLAIN Output Format}. URL: \url{http://dev.mysql.com/doc/refman/5.6/en/explain-output.html}, Abruf am 28.6.2014
	\bibitem{refman3}ORACLE MySQL Documentation (2014): {\sl EXPLAIN EXTENDED Output Format}. URL:
	\url{http://dev.mysql.com/doc/refman/5.6/en/explain-extended.html}, Abruf am 28.6.2014

  \end{thebibliography}
