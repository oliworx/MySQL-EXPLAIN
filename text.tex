
\section{Einleitung}

\subsection{Problemstellung}
Datenbank-Systeme finden heute in nahezu allen IT-Systemen Verwendung.
Der Optimierung von Datenbank-Anfragen kommt daher eine große Bedeutung zu.
Hierfür gibt es eine Vielzahl von Möglichkeiten, z.B. Latenz und Bandbreite der Anbindung der Datenbank, Leistungsfähigkeit des Datenbank-Servers, Anzahl der Datenbank-Anfragen im Programmcode, Cachingmeachanismen.

Hat man andere Flaschenhälse ausgeschlossen oder bereits optimiert,  gilt es die relevanten SQL-Abfragen des Systems zu identifizieren und gezielt zu optimieren.

Viele RDBMS stellen mit dem SQL-Kommando EXPLAIN eine Möglichkeit zur Verfügung, mehr über die innere Arbeitsweise der Datenbank bei einer bestimmten SQL-Abfrage zu erfahren.
Durch gezielte Veränderung der SQL-Abfrage oder des Datenschemas kann somit die Bearbeitung der Abfrage optimiert werden.

\subsection{Zielsetzung der vorliegenden  wissenschaftlichen  Auseinandersetzung}
% Eine Semesterarbeit in diesem Bereich soll sich mit MySQL spezifischen Fragestellungen auseinander setzen. Es sollen auch die theoretischen Grundlagen der zu behandelnden datenbanktheoretischen Ansätze erläutern.
Die folgende Arbeit bezieht sich speziell auf die Optimierung von SQL-Anfragen mitttels EXPLAIN bei dem RDBMS MySQL.
Es soll untersucht werden

\subsection{Vorgehensbeschreibung}
-Literaturrecherche betrieben,
-wesentliche Punkte zusammengefasst
-an Beispieldatenbank experimentell nachvollzogen

\section{Theoretische Grundlagen}
\subsection{Der physische Zugriff auf die Daten}
Daten einer DB werden in der Regel auf einer Festplatte (HDD) oder einem Flash-Laufwerk (SSD) gespeichert.
Das RDBMS nutzt dazu Funktionen des Betriebssystems auf verschiedenen Ebenen.
Dateisystem-Treiber, nimmt Lese und Schreibanforderungen für Datensätze an und gibt rechnet diese in nie durchnumerierten Blöcke des Blockgerätes um. Der Blockgeräte-Treiber liest dann die entsprenden Blöcke von der Platte oder schreibt sie.

DBMS -> Dateisystemtreiber -> Blockgerätetreiber -> HDD/SSD 

 HDD-Blockgerätetreiber-Dateisystemtreiber-DBMS

\subsection{Speicherstrukturen}
 Binärbaum:

 B-Baum, Hashing, Heap

\subsection{Bearbeitung von SQL-Statements}
 Umsetzung in relationale Algebra

\subsection{Optimierungen : frühzeitige Restriktionen, JOINs}


\section{Beispiel MySQL}

\subsection{einfache Select-Anfragen (eine Tabelle)}

\subsection{Umschreiben von Nicht-Select-Anfragen}
As of MySQL 5.6.3, permitted explainable statements for EXPLAIN are SELECT, 
DELETE, INSERT, REPLACE, and UPDATE. Before MySQL 5.6.3, SELECT is the only explainable statement.
\cite{refman1}

\subsection{Die Spalten der EXPLAIN-Ausgabe}
id\\*
identifiziert das SELECT, zu dem die Zeile gehört, bei einfachen SELECT immer nur 1

select\_type\\*
einfaches oder komplexes SELECT
\begin{description}
	\item[SIMPLE] Einfaches select, keine Unterabfragen oder UNIONS
	\item[PRIMARY] äußeres eines komplexen SELECT
	\item[SUBQUERY] Select in einer Unterabfrage
	\item[DERIVED] SELECT in Unterabfrage in FROM-Klausel
	\item[UNION] zweites bzw. nachfolgende SELECT einer Union
	\item[UNION RESULT] Ergebnis des UNION, wird aus temporärer Tabelle geholt.
\end{description}

table\\*
auf welche Tabelle wird zugegriffen
\begin{description}
	\item Tabellenname oder Alias
	\item Spalte von oben nach unten lesen
\end{description}

type:
Wie ist der Zugriffstyp, wie wird MySQL die Zeilen in der Tabelle auffinden?
-ALL = Tablescan, Tabelle muss in der Regel von Anfang bis Ende durchlaufen werden
-index = Wie Tablescan, aber in Indexreihenfolge. Sortierung wird vermieden.
-range=Bereichsscan= eingeschränkter Indexscan, z.B bei BETWEEN oder WHERE x >
-ref = Indexzugriff, Index-Lookup, Zeilen entsprechen einem Wert, nur bei einem nichteindeutigen Index, Index wird mit einem Referenzwert verglichen. Variante: ref\_or\_null
-eq\_ref = Index-Lookup mit eindeutigem Treffer, bei Primärschlüssel oder eindeutigem Index
-const,system = konnte von MySQL wegoptimiert oder in eine Konstante umgewandelt werden.
-NULL = Abfrage kann von MySQL bei der Optimierung aufgelöst werden, kein Zugriff auf Tabelle oder Index, z.B. Minimum einer indizierten Spalte

possible\_keys:
-welche Indizes kömnnten benutzt werden
-frühe Phase der Optimierung
-können später nutzlos sein

key:
-welchen Index wählt der Optimierer für die Abfrage
-auch abdeckende Indizes

key\_len
-wie viel Byte (Spaltenbreite) eines Index werden benutzt
-welche Spalten des Index werden genutzt, von links beginnend

ref
-welche Spalten aus früheren Tabellen werden benutzt, um in dem key-Index nachzuschlagen

rows
-Schätzung, Anzahl der Zeilen, die gelesen werden müssen
-pro schleife im Nested-Loop-Join-Plan
-Schätzung kann ungenau sein

filtered
-neu seit MySQL 5.1
-bei Explain EXTENDED
-pessimistische Schätzung des Prozentsatzes der Zeilen

Extra
-Using Index = abdeckender Index benutzt
-Using where = Zeilen nachträglich gefiltert, also nicht den Index für die WHERE-Bedingung genutzt
-Using temporary = temporäre Tabelle für Sortierung
-Using filesort = externe Sortierung, im RAM oder auf HDD
-ranke checked for each record (index map: N) = kein guter Index vorhanden
EXPLAIN Output Format \cite{refman2}

\subsubsection{EXPLAIN EXTENDED}
\subsubsection{EXPLAIN PARTITIONS}
\subsection{Abfragen mit mehreren Tabellen}
\subsection{Optimierungsmöglichkeiten und Benchmarking}
\subsection{Visuelles EXPLAIN (graphische Werkzeuge)}

\section{Fazit und Ausblick}
% Der  Schluss  stellt  quasi  die  Abrundung  der  Arbeit  dar.  Er  beinhaltet  eine  kurze  Zusammenfassung
% der wichtigsten Ergebnisse, ein Fazit sowie einen Ausblick auf weitere Fragestellungen bzw. künftige Entwicklungen

Beschränkungen!
Optimierung wichtig
Mit Explain möglich
nicht immer exakte Angaben
Kontrolle der Optimierung mit Benchmarks nötig
möglichst bereits in den Entwicklungsprozess integrieren, und nicht erst wenn es brennt
