
\section{Einleitung}
\subsection{Problemstellung}
Datenbank-Systeme finden heute in nahezu allen IT-Systemen Verwendung.
Der Optimierung von Datenbank-Anfragen kommt daher eine große Bedeutung zu.
Hierfür gibt es eine Vielzahl von Möglichkeiten, z.B. Latenz und Bandbreite der Anbindung der Datenbank, Leistungsfähigkeit des Datenbank-Servers, Anzahl der Datenbank-Anfragen im Programmcode, Cachingmeachanismen.

Hat man andere Flaschenhälse ausgeschlossen oder bereits optimiert,  gilt es die relevanten SQL-Abfragen des Systems zu identifizieren und gezielt zu optimieren.

Viele RDBMS stellen mit dem SQL-Kommando EXPLAIN eine Möglichkeit zur Verfügung, mehr über die innere Arbeitsweise der Datenbank bei einer bestimmten SQL-Abfrage zu erfahren.
Durch gezielte Veränderung der SQL-Abfrage oder des Datenschemas kann somit die Bearbeitung der Abfrage optimiert werden.

\subsection{Zielsetzung der vorliegenden  wissenschaftlichen  Auseinandersetzung}
% Eine Semesterarbeit in diesem Bereich soll sich mit MySQL spezifischen Fragestellungen auseinander setzen. Es sollen auch die theoretischen Grundlagen der zu behandelnden datenbanktheoretischen Ansätze erläutern.
Die folgende Arbeit bezieht sich speziell auf die Optimierung von SQL-Anfragen mitttels EXPLAIN bei dem RDBMS MySQL.
Es soll untersucht werden

\subsection{Vorgehensbeschreibung}


\section{Theoretische Grundlagen}
\subsection{Der physische Zugriff auf die Daten}
 HDD-Blockgerätetreiber-Dateisystemtreiber-DBMS
 
\subsection{Speicherstrukturen}
 Binärbaum, B-Baum, Hashing, Heap
 
\subsection{Bearbeitung von SQL-Statements}
 Umsetzung in relationale Algebra
 
\subsection{Optimierungen : frühzeitige Restriktionen, JOINs}


\section{Beispiel MySQL}
\subsection{einfache Select-Anfragen (eine Tabelle)}
\subsection{Umschreiben von Nicht-Select-Anfragen}
\subsection{Die Spalten der EXPLAIN-Ausgabe}
\subsection{EXPLAIN EXTENDED}
\subsection{EXPLAIN PARTITIONS}
\subsection{Abfragen mit mehreren Tabellen}
\subsection{Optimierungsmöglichkeiten und Benchmarking}
\subsection{Visuelles EXPLAIN (graphische Werkzeuge)}

\section{Fazit und Ausblick}
% Der  Schluss  stellt  quasi  die  Abrundung  der  Arbeit  dar.  Er  beinhaltet  eine  kurze  Zusammenfassung
% der wichtigsten Ergebnisse, ein Fazit sowie einen Ausblick auf weitere Fragestellungen bzw. künftige Entwicklungen

Beschränkungen!
Optimierung wichtig
Mit Explain möglich
nicht immer exakte Angaben
Kontrolle der Optimierung mit Benchmarks nötig
möglichst bereits in den Entwicklungsprozess integrieren, und nicht erst wenn es brennt
